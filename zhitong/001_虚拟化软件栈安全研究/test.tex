%-*- coding: UTF-8 -*-
% 论文总结.tex
% 2020年7月第一周
\documentclass[UTF8]{ctexart}
\usepackage{graphicx}
\usepackage{float}
\usepackage{amsmath}
\usepackage{geometry}
\geometry{a4paper,centering,scale=0.8}
\usepackage[format=hang,font=small,textfont=it]{caption}
\usepackage[nottoc]{tocbibind}
\usepackage{url}
\usepackage{listings}
\usepackage{booktabs}
\usepackage{xcolor}     %高亮使用的颜色
\definecolor{commentcolor}{RGB}{85,139,78}
\definecolor{stringcolor}{RGB}{206,145,108}
\definecolor{keywordcolor}{RGB}{34,34,250}
\definecolor{backcolor}{RGB}{220,220,220}

\lstset{
 columns=fixed,       
 numbers=left,                                        % 在左侧显示行号
 numberstyle=\tiny\color{gray},                       % 设定行号格式
 frame=none,                                          % 不显示背景边框
 backgroundcolor=\color[RGB]{245,245,244},            % 设定背景颜色
 keywordstyle=\color[RGB]{40,40,255},                 % 设定关键字颜色
 numberstyle=\footnotesize\color{darkgray},           
 commentstyle=\it\color[RGB]{0,96,96},                % 设置代码注释的格式
 stringstyle=\rmfamily\slshape\color[RGB]{128,0,0},   % 设置字符串格式
 showstringspaces=false,                              % 不显示字符串中的空格
 language=c++,                                        % 设置语言
}

\newenvironment{myquote}
{\begin{quote}\kaishu\zihao{-5}}
{\end{quote}}

\newcommand\degree{^\circ}

\title{\heiti 虚拟化软件栈安全研究(计算机学报-中科院信工所-朱民)}
\author{\kaishu 至彤}
\date{\today}

\bibliographystyle{plain}

\newtheorem{thm}{定理}

\begin{document}
    
    \maketitle

    \clearpage
    \section{摘要}\label{sec:diyijie}
	在虚拟化软件栈,虚拟机监控器具有最高权限和较小的可信计算基,故而能为虚拟化系统提供安全监控和保护.但同时也引入了新的软件层,增加了脆弱性,增大了攻击面.另外,多租户模式以及软硬件平台资源共享,更加剧了新软件栈的安全威胁。
	\begin{itemize}
	\item[*] 分析虚拟化软件栈的安全威胁、攻击方式和威胁机理
	\item[*] 比较了国内外相关安全方案和技术,并指出了当前仍然存在的安全问题.
	\item[*] 对未来的研究方向进行了探讨和分析,给出了虚拟化软件栈的安全增强方案。
	\end{itemize}
	\clearpage
	
    \section{引言}\label{sec:dierjie}
	\begin{itemize}
	\item[*] 虚拟化扩增了传统服务器的软件栈.软件栈越大、越复杂,攻击面和脆弱性就越多,安全性则更难以保障。
	\item[*] 虚拟化技术提供的隔离性并不强
	\item[*] 软件漏洞,攻击者利用侧信道攻击也可窃取其它虚拟机的敏感数据
	\item[*] 当前虚拟化的研究主要集中在对Hypervisor的保护、对虚拟机的隔离以及对VM的内部系统、应用的保护,甚至将虚拟化从可信计算基中剔除,以此来增强虚拟化软件栈的安全.
	\end{itemize}
	\clearpage
	
	\bibliography{test}
\end{document}

