%-*- coding: UTF-8 -*-
% 论文总结.tex
% 2020年7月第二周
\documentclass[UTF8]{ctexart}
\usepackage{graphicx}
\usepackage{float}
\usepackage{amsmath}
\usepackage{geometry}
\geometry{a4paper,centering,scale=0.8}
\usepackage[format=hang,font=small,textfont=it]{caption}
\usepackage[nottoc]{tocbibind}
\usepackage{url}
\usepackage{listings}
\usepackage{booktabs}
\usepackage{xcolor}     %高亮使用的颜色

\definecolor{commentcolor}{RGB}{85,139,78}
\definecolor{stringcolor}{RGB}{206,145,108}
\definecolor{keywordcolor}{RGB}{34,34,250}
\definecolor{backcolor}{RGB}{220,220,220}

\lstset{
 columns=fixed,       
 numbers=left,                                        % 在左侧显示行号
 numberstyle=\tiny\color{gray},                       % 设定行号格式
 frame=none,                                          % 不显示背景边框
 backgroundcolor=\color[RGB]{245,245,244},            % 设定背景颜色
 keywordstyle=\color[RGB]{40,40,255},                 % 设定关键字颜色
 numberstyle=\footnotesize\color{darkgray},           
 commentstyle=\it\color[RGB]{0,96,96},                % 设置代码注释的格式
 stringstyle=\rmfamily\slshape\color[RGB]{128,0,0},   % 设置字符串格式
 showstringspaces=false,                              % 不显示字符串中的空格
 language=c++,                                        % 设置语言
}

\newenvironment{myquote}
{\begin{quote}\kaishu\zihao{-5}}
{\end{quote}}

\newcommand\degree{^\circ}

\title{\heiti Houdini’s Escape: Breaking the Resource Rein of Linux Control Groups}
\author{\kaishu 至彤}
\date{\today}

\bibliographystyle{plain}

\newtheorem{thm}{定理}

\begin{document}
    
    \maketitle

    \clearpage
    \section{摘要}\label{sec:diyijie}
	LCG,也叫cgroups是操作系统级容器化的基础,cgroups机制将进程划分成有层次的进程组,并且应用不同的控制器来管理系统资源(CPU,内存,块IO),子进程会自动复制父进程的cgroups属性进行资源控制。但是这种情况下,子进程的cgroups限制和资源记账(resource accounting)在特殊情况下会失效。
	\clearpage

    \section{基础}\label{sec:dierjie}
    \subsection{容器资源隔离}
    容器资源隔离和管理基础是Linux Namspaces和Linux Control Groups,容器部署安全的机制包括Capabilities,SELinux,AppArmor,Seccomp和Security Namespace。
    \subsection{cgroups层次和controller}
    cgroups是树状结构,线程只能关联至每个层次结构的一个cgroups,但是课关联至多个层次结构,每个层次结构有多个子系统关联。所以cgroups机制能限制进程组的总资源。
    \subsubsection{cpu controller}
    \subsubsection{cpusets controller}
    \subsubsection{blkio  controller}
    \subsubsection{pid controller}
    \subsection{cgroups继承性}
    Fork或Clone的子进程和父进程的cgroups相同
	\clearpage

	\section{Exploiting策略生成额外工作负载}\label{sec:disnajie}
	\subsection{内核Upcalls}
	利用内核线程启动新进程,避免cgroups资源限制。在用户空间的进程启动内核方法(在内核空间),比如用户空间的系统调用例子:usemode helper API
	\subsection{内核线程}
	例子:kthreadd,kworker,ksoftirqd,migration,kswapd
	\subsection{服务进程}
	服务进程:进程管理,系统信息日志,debug信息
	\subsection{中断上下文}
	硬中断,软中断softirqs(ksoftirqd)
	\clearpage
	    
	\section{案例分析}\label{sec:diwujie}
	5个案例分析现实情况下Docker 容器破坏cgroups的资源限制,进一步在多租户的容器环境下放大资源消耗从而攻击其他容器。
	\subsection{异常处理}
	调用usermode helper API,通过异常触发用户空间进程,消耗200倍CPU资源,降低同宿主机容器性能85%-95%
	\subsection{数据同步}
	磁盘数据同步的writeback机制,系统调用:sync,syncfs,fsync,其中sync能降低系统级别的IO性能,造成Resource-Freeing Attack (RFA)和隐藏信道
	\subsection{系统进程-Journald}
	内核日志,系统日志与审计记录有系统进程journald记录,su(切换用户命令),添加用户或用户组,exception
	\subsection{容器引擎}
	增加内核线程(kworker)和容器引擎的工作负载,资源消耗可高达三倍
	\subsection{softirq处理}
	\subsubsection{NET softirq—NIC在包传输后触发,可被iptables放大}
	\subsubsection{Block softirq--块设备IO中断}
	\clearpage
	\section{实验}\label{sec:diwujie}
	在Amazon EC2云环境下进行实验,容器可以消耗200倍以上的资源,减少百分之九十五co-resident 容器的计算和IO资源。
	\clearpage

	\section{相关工作-容器安全}\label{sec:diliujie}
	\begin{itemize}
	\item[*] 系统调用限制 [\begin{upshape} \itshape Speaker: Split-Phase Execution of Application Containers\end{upshape}]
	\item[*] 基于内存的伪文件系统攻击 [\begin{upshape} \itshape A Study on the Security Implications of Information Leakages in Container Clouds \end{upshape}]
	\item[*] Namespaces安全策略冲突处理 [\begin{upshape} \itshape Security Namespace: Making Linux Security Frameworks Available to Containers\end{upshape}]
	\item[*] 利用Intel SGX 构建secure Linux containers  [\begin{upshape} \itshape SCONE: Secure Linux Containers with Intel SGX][SCONE: Secure Linux Containers with Intel SGX \end{upshape}]
	\item[*] 容器内的隐藏信道攻击  [\begin{upshape} \itshape Whispers between the Containers: High-Capacity Covert Channel Attacks in Docker \end{upshape}]
	\end{itemize}
	\clearpage
	
	\section{相关工作-云安全}\label{sec:diqijie}
	\begin{itemize}
	\item[*] Co-residence	
		\begin{itemize}
		\item[·] 侧信道攻击 [\begin{upshape} \itshape Exploring Information Leakage in Third-Party Compute Clouds \end{upshape}]
		\item[·] 隐藏信道攻击 [\begin{upshape} \itshape 4k-Aliasing Covert Channel and Multi-Tenant Detection in IaaS Clouds \end{upshape}]
		\item[·] 资源计费问题 [\begin{upshape} \itshape Peeking behind the curtains of serverless platforms\end{upshape}]
		\end{itemize}
	\item[*] DOS攻击
		\begin{itemize}
		\item[·] 内存,IO攻击 [\begin{upshape} \itshape An Experimental Study of Cascading Performance Interference in a Virtualized Environment \end{upshape}]
		\item[·] resource-freeing攻击 [\begin{upshape} \itshape Resource-Freeing Attacks: Improve Your Cloud Performance  \end{upshape}]
		\end{itemize}
	\end{itemize}
	\clearpage
	
	\bibliography{test}
\end{document}

